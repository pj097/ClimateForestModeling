\chapter{Analysis and Discussion}
\label{chapter:analysis}

\section{Seasonal Analysis}

Fig.\,\ref{fig:seasonal_selection} shows the performance of the CNN model across different seasons, with metrics including recall, precision, weighted f1-score, precision-recall curve area under the curve (PRC), and receiver operating characteristic area under the curve (AUC). Filled boxes in the bottom plot of Fig.\,\ref{fig:seasonal_selection} indicate the seasons used to train and validate the model. Blue boxes indicate a combination of 3D CNN and fully-connected layers and the orange box indicates the use of a similar model but with the introduction Long Short-Term Memory (LSTM) alongside 3D convolutions and fully-connected layers.

\begin{figure}[ht]
    \centering
    \includegraphics[width=0.9\linewidth, trim={20pt 40pt 10pt 30pt}, clip]{figures/figures_analysis/seasonal_selection.pdf}
    \caption{Seasonal Sentinel-2 analysis for all season combinations using a 3D CNN.}
    \label{fig:seasonal_selection}
\end{figure}


3D convolutions were selected for this model due to their suitability for this problem as they can effectively capture the spatial and spectral dependencies in the multi-temporal Sentinel-2 data. By considering the additional temporal dimension, 3D convolutions can leverage seasonal variations and changes in vegetation phenology, which are crucial for accurate tree genus classification.

LSTM was introduced to the model alongside 2D convolutions because they process the spatial structure of the Sentinel-2 images through convolution operations while simultaneously capturing temporal sequences. This dual capability allows the model to learn intricate spatial patterns within each image and understand how these patterns evolve over time.

The selected metrics used to create Fig.\,\ref{fig:seasonal_selection} are well-suited for handling class imbalance in the seasonal analysis of the CNN model. Recall ensures the model captures as many instances of the minority classes as possible, which is critical when dealing with imbalanced datasets. Precision assesses the accuracy of the model's positive predictions, reducing the impact of false positives. The weighted f1-score balances precision and recall, accounting for class imbalance by considering the support of each class. PRC focuses on the trade-off between precision and recall, highlighting the model's performance on minority classes. Lastly, AUC measures overall model performance across all thresholds, providing a comprehensive view of its ability to distinguish between classes.

For individual seasons, the model performs best in summer and autumn overall across the metrics. This suggests that the CNN model is more effective at classifying tree genera during these times, likely due to clearer and more distinct spectral signatures in the data collected during summer and autumn. The lower performance in spring and winter might be attributed to less distinct spectral signatures or more challenging environmental conditions, such as cloud cover and snow, which can affect data quality.

Based on the weighted f1-scores shown in Fig.\,\ref{fig:seasonal_selection}, adding more seasons does not seem to offer significant benefits. For instance, some two-season combinations, such as summer and autumn, performed on par with the more complex four-season models. Additionally, single-season models were only a few percentage points below the top-performing models.

Based on these results, further analysis focused solely on summer seasons. This approach benefits from faster model training and reduced storage requirements, as adding an extra season nearly doubles the storage needs, a challenge that intensifies with the extension of the analysis over additional years. Despite these adjustments, a complete Sentinel-2 dataset for a single season still requires nearly 200GB, or approximately  1MB per location.

\section{Sentinel-2 Band Analysis}

