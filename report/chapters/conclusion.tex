\chapter{Conclusions and Future Directions}
\label{chapter:conclusion}
\section{Conclusions}


This study explored how various factors influence tree genus classification using Sentinel-2 imagery, different seasonal and spectral band combinations, and additional environmental datasets.

The analysis revealed that the classification model performed most effectively during summer and autumn, when the spectral signatures for tree genera were the clearest and most distinct. In contrast, performance dropped during spring and winter, likely due to less distinct spectral features and complications from environmental factors such as cloud cover and snow.

Among the Sentinel-2 bands, those in the NIR and SWIR ranges yielded better results. The combination of bands B3, B6, B8, and B11 emerged as the optimal choice, balancing classification accuracy with storage efficiency. This blend provided an effective trade-off between data richness and computational demands.

Incorporating soil composition and elevation data offered only marginal improvements in model performance compared to using Sentinel-2 data alone. The high-resolution spectral data from Sentinel-2 proved to be the most critical factor in achieving accurate classification results.

Further analysis using the ERA5 dataset revealed that short-term climate variations had minimal impact on tree genus classification over a 5-year period. This brief perspective may not fully capture long-term climate effects, which could significantly influence ecological dynamics. Additionally, focusing on Europe, a region currently experiencing relatively moderate climate changes, might limit the broader applicability of these findings.

Efforts to model the relationship between climate change and tree genus distributions using a regression neural network were unsuccessful. The model demonstrated poor performance, with low R$^2$ values and high MAE, indicating that the climate predictors did not effectively explain the variations in tree genus changes.

In conclusion, this research underscores the critical role of seasonal timing and spectral band selection in tree genus classification. It also highlights the limited impact of additional environmental data and short-term climate changes on model performance. While the current models can achieve high accuracy with well-chosen data and parameters, predicting the long-term impacts of climate change on tree distributions remains a complex and challenging task.

\section{Future Directions}

To gain a more comprehensive understanding of the effects of climate change on tree genera, future research should focus on long-term data analysis. This approach involves examining extended time periods and incorporating broader climate datasets to capture gradual shifts and potential delayed effects on tree distributions. While Sentinel-2 data is well-suited for short-term studies, other datasets like Landsat, which extends back several decades, offer valuable historical context.

Expanding the geographic scope of the study to include regions outside Europe could provide more generalized insights. Areas experiencing more pronounced climate changes might reveal different patterns and interactions between climate variables and tree genera or species distributions.

Future research should also consider differentiating between commercial forests and long-term, natural forests. The change maps of these two forest types may differ significantly due to their distinct management practices and ecological dynamics. Commercial forests are often subject to planned harvesting, replanting, and other management activities, which can create distinct, predictable patterns in their change maps. In contrast, long-term, natural forests undergo changes driven by more complex ecological processes, such as natural disturbances, successional dynamics, and gradual environmental shifts. Analyzing these differences could provide deeper insights into how various forest management strategies and natural processes influence tree genus distributions and responses to climate change, leading to more nuanced and actionable findings.

Additionally, future research could benefit from incorporating additional environmental variables beyond soil composition and elevation. Variables such as land cover changes, pest outbreaks, and other biotic factors may significantly influence tree genus distributions and could be integrated into a more comprehensive model.

Lastly, collaborating with ecologists and botanists could enhance the study by incorporating biological factors that affect tree distributions. Gaining insights into ecological processes and species-specific responses to environmental changes can help refine model parameters and improve prediction accuracy.