\chapter{Methodology}

\section{Develop Methods for Tree Species Predictions}

\begin{itemize}
  \item \textbf{Data collection:} gather multi-temporal, multi-spectral satellite imagery covering the study area. Obtain ground-truth data on tree species composition through existing forest inventory datasets.

  \item \textbf{Data exploration:} visualise the data and use statistical techniques to describe dataset characterisations, such as size, quantity, and accuracy, in order to better understand the nature of the data.
  
  \item \textbf{Pre-processing:} pre-process the satellite imagery to remove noise, correct for atmospheric effects, and normalise.

  
  \item \textbf{Model training:} train machine learning algorithms, such as Convolutional Neural Network (CNN) and Long Short-Term Memory (LSTM), using the pre-processed data and ground-truth labels to develop accurate prediction models.
\end{itemize}

\section{Detect Changes in Forest Cover}

\begin{itemize}
  \item \textbf{Time-series analysis:} obtain multi-temporal satellite imagery covering multiple time periods to capture changes in forest cover over time.
  
  \item \textbf{Change detection algorithm:} if feasible, implement change detection algorithms to identify areas of forest cover change between consecutive time steps.
\end{itemize}

\section{Integrate Climate Data}

\begin{itemize}
  \item \textbf{Climate data acquisition:} subject to the quality of previous results, obtain climate data, such as temperature, precipitation, and soil moisture, from gridded climate datasets covering the study area or global averages.
  
  \item \textbf{statistical analysis:} perform statistical analysis to assess the relationships between climate variables and tree species dynamics, using techniques such as correlation analysis or regression modelling.
  
\end{itemize}