%Two resources useful for abstract writing.
% Guidance of how to write an abstract/summary provided by Nature: https://cbs.umn.edu/sites/cbs.umn.edu/files/public/downloads/Annotated_Nature_abstract.pdf %https://writingcenter.gmu.edu/guides/writing-an-abstract
\chapter*{\center \Large  Abstract}
%%%%%%%%%%%%%%%%%%%%%%%%%%%%%%%%%%%%%%
% Replace all text with your text
%%%%%%%%%%%%%%%%%%%%%%%%%%%%%%%%%%%
This study investigates tree genus classification using Sentinel-2 imagery, with a focus on seasonal and spectral band combinations, alongside environmental data integration. The analysis revealed the model performed best in summer and autumn, when clearer spectral signatures enhanced accuracy. The optimal spectral band combination—B3, B6, B8, and B11—balanced classification performance and storage efficiency. Incorporating soil composition and elevation data provided only marginal improvements, with Sentinel-2’s high-resolution spectral data being the most influential factor. Climate data from ECMWF Reanalysis v5 (ERA5) showed minimal Pearson correlations with classification change maps over a five-year period. Regression neural network models attempting to assess climate change effects yielded poor results, indicating no discernible impact on dominant European tree genera. Future research should focus on long-term data to better understand gradual climate effects, explore regions experiencing more pronounced climate changes, and distinguish between commercial and natural forests to identify different change patterns. Incorporating additional environmental variables and collaboration with ecologists could further improve model accuracy and insights into tree genus distribution.

%%%%%%%%%%%%%%%%%%%%%%%%%%%%%%%%%%%%%%%%%%%%%%%%%%%%%%%%%%%%%%%%%%%%%%%%%s
~\\[1cm]
\noindent % Provide your key words
\textbf{Keywords:} Convolutional Neural Network (CNN), Classification, Regression, Sentinel-2, Copernicus, SoilGrids, EU-Forest, Forests, Tree Genus, Climate, ERA5

\vfill
\noindent
\textbf{Report's total word count:} 6000-7000