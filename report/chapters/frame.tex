\chapter{Research Framework}
\label{chapter:frame}
\section{Research Question}

Can we generate tree species predictions and change maps using remote sensing data and existing ground-truth data to assist in mitigating the impacts of climate change on forest ecosystems?

\section{Justification}

Forests are vital for maintaining biodiversity, regulating the climate, and providing essential ecosystem services. However, they are increasingly vulnerable to the effects of climate change, including shifts in temperature and precipitation patterns, increased frequency and intensity of extreme weather events, and altered disturbance regimes. Understanding how forests are responding to these changes is crucial for effective conservation and management efforts.

By developing methods to generate tree species predictions and change maps from remote sensing data, this research will attempt to address a pressing need for tools to monitor and assess the impacts of climate change on forest ecosystems. Such maps can provide valuable insights into changes in species composition, distribution, and health over time, helping to identify areas of conservation concern, prioritise management interventions, and inform climate adaptation strategies.

\section{Research Objectives}

\begin{itemize}
\item Establish accurate prediction models using remote sensing data and ground-truth labels.
\item Identify changes in forest cover over time, contingent on the success of the tree species prediction models.
\item Analyse the influence of climate variables on forest dynamics, depending on the quality of tree species predictions and change detection results.
\item Summarise findings and propose further research or practical applications.
\end{itemize}

\section{Measurement of Achievements}

\begin{itemize}
\item Evaluate the accuracy and reliability of tree species prediction and change detection algorithms.
\item Assess the ability to effectively integrate climate data and analyse its influence on forest dynamics, conditional on the quality of initial results.
\item Determine how useful the generated maps are for informing conservation and management decisions.
\item Suggest next steps and areas for further investigation based on the study's findings.
\end{itemize}

