\chapter{Introduction}
\label{chapter:intro}

Forests play a critical role in regulating the Earth's climate and supporting biodiversity, as emphasized by studies such as \cite{bonan2008forests} and \cite{watson2018intact}. However, they are increasingly threatened by human activities and environmental changes, including deforestation, land-use conversion, and climate change, which are well-documented in \cite{fao2020gfra} and \cite{hansen2013forestcover}. Understanding the composition and dynamics of forest ecosystems is essential for effective conservation and management efforts, as discussed in \cite{turner2003remotesensing}. Accurate mapping of tree species and monitoring changes in forest cover over time, as demonstrated by \cite{sexton2013treecover}, are fundamental tasks in ecosystem management. Such maps provide valuable information for assessing biodiversity, tracking habitat loss, and understanding the impacts of climate change on forest ecosystems, as detailed by \cite{vose2018forests}.

\section{Remote Sensing and Machine Learning in Forest Monitoring}

Remote sensing technologies, including satellite imagery and LiDAR data, have revolutionized the ability to map and monitor forests at regional and global scales. Satellite imagery has provided comprehensive coverage and frequent updates, while LiDAR data offers precise measurements of forest structure, as shown in \cite{pettorelli2016satellite} and \cite{lefsky2002lidar}. Recent advancements in machine learning algorithms, particularly Convolutional Neural Networks (CNNs) and Random Forests (RF), have significantly enhanced the accuracy of tree species predictions from remotely sensed data (\cite{zheng2019deep, breiman2001random}). This study aims to develop and evaluate methods for generating tree genus predictions and change maps using these advanced remote sensing technologies. By leveraging these innovations, this study aims to advance our understanding of forest responses to environmental changes. Building on the methodologies and findings of recent studies such as \cite{hansen2013forestcover}, \cite{belgium_classification}, \cite{pakistan}, \cite{copernicus_main}, and \cite{germany_bavaria}, this research seeks to refine and enhance predictive models and change maps, offering new insights into how forests adapt to and are affected by shifting environmental conditions.

\section{Data Integration and Methodology}

In this research, high-resolution Sentinel-2 imagery has been integrated with detailed EU-Forest data (\cite{eu_forest_data}), Copernicus DEM elevation information (\cite{dem_dataset}), SoilGrids soil properties (\cite{soil_report}), and ECMWF Reanalysis v5 (ERA5) climate data (\cite{era5_dataset}). This combination of datasets provides a robust foundation for developing a CNN classifier. The classifier is designed to accurately identify tree genera across diverse European landscapes. Sentinel-2 imagery provides detailed multispectral data at a 10-meter and 20-meter resolutions, capturing the intricate spectral signatures of vegetation. The EU-Forest dataset offers essential ground-truth data across about 250,000 locations in Europe, facilitating the training and validation of machine learning models. Elevation data from the Copernicus DEM GLO-30 dataset introduces topographical context that influences vegetation distribution, while SoilGrids provides comprehensive soil composition information crucial for understanding the ecological niches of various tree genera.

Furthermore, climate variables, including temperature, precipitation, and soil moisture, have been integrated into the analysis to model the relationships between environmental factors and vegetation dynamics. This integration aims to provide deeper insights into how climate change is affecting forest ecosystems and to enhance the predictive power of the classification models. By combining these diverse datasets, the study endeavors to create a more nuanced and reliable model for tree genus classification.

\section{Challenges and Considerations}

Detecting changes in forest composition is essential for assessing the impacts of disturbances such as wildfire, climate change, and tree diseases. Time-series analysis of satellite imagery, combined with advanced algorithms for change detection, can enable researchers to quantify forest dynamics and identify areas undergoing significant ecological transitions. The inclusion of climate data further enriches this analysis by allowing for the exploration of how shifting environmental conditions are influencing forest ecosystems over time.

Despite these efforts, several challenges persist in remote sensing-based tree species classification and change mapping. These include the need for improved methods to handle the complexity of forest ecosystems, increasing the availability of ground-truth data, incorporating uncertainty into mapping algorithms, and scaling up analyses to cover larger geographical areas. Additionally, the addition of elevation, soil, and climate data in this study resulted in only marginal improvements in model accuracy, prompting a critical evaluation of the relative importance of spectral versus environmental data in tree genus classification. This outcome suggests that while these additional datasets provide valuable context, their practical impact on classification accuracy within this specific framework may be limited.

This research highlights the complexities and trade-offs involved in integrating multi-source data into remote sensing models for forest monitoring. The findings underscore the importance of continued exploration and refinement of methodologies to optimize the use of diverse environmental datasets in understanding and conserving forest ecosystems in the face of global environmental change.