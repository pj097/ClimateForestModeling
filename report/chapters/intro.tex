\chapter{Introduction}
\label{chapter:intro}
\section{Background}

Forests play a critical role in regulating the Earth's climate and supporting biodiversity. However, they are increasingly threatened by human activities and environmental changes, including deforestation, land-use conversion, and climate change. Understanding the composition and dynamics of forest ecosystems is essential for effective conservation and management efforts.

Accurately mapping tree species and monitoring changes in forest cover over time are fundamental tasks in ecosystem management and conservation. Such maps provide valuable information for assessing biodiversity, tracking habitat loss, and understanding the impacts of climate change on forest ecosystems.

This study aims to develop and evaluate methods for generating tree species predictions and change maps using remote sensing data. By leveraging advances in machine learning and image processing techniques, the project seeks to contribute to our understanding of how forests are responding to environmental changes and inform strategies for mitigating the effects of climate change.

\section{Literature Review}

Remote sensing technologies, including satellite imagery and LiDAR data, have revolutionised our ability to map and monitor forests at regional and global scales. Recent developments in machine learning algorithms, such as Convolutional Neural Networks (CNNs) and Random Forests (RF), have significantly improved the accuracy of tree species predictions from remotely sensed data (\cite{belgium_classification, copernicus_main, germany_bavaria}).

Detecting changes in forest compositions is essential for assessing the impacts of disturbances such as wildfire, climate change, and tree diseases (\cite{usda}). Time-series analysis of satellite imagery, combined with advanced algorithms for change detection, can enable researchers to quantify forest dynamics and identify areas undergoing significant ecological transitions (\cite{indonesia}).

Climate variables, including temperature, precipitation, and soil moisture, influence the distribution and health of tree species. Integrating climate data with remote sensing imagery allows researchers to model the relationships between environmental factors and vegetation dynamics, providing insights into how climate change is altering forest ecosystems (\cite{pakistan}).

While progress has been made in remote sensing-based tree species classification and change mapping, several challenges remain. These include the need for improved methods for handling complex forest ecosystems, increase in the availability of ground-truth data, incorporating uncertainty into mapping algorithms, and scaling up analyses to cover larger geographical areas.

